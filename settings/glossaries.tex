\usepackage[acronym, toc]{glossaries}

\makeglossaries

% Example
% \newglossaryentry{maths}
% {
%         name=mathematics,
%         description={Mathematics is what mathematicians do}
% }

\newacronym{cpu}{CPU}{Central Processing Unit}
\newacronym{soc}{SoC}{System on a chip}
\newacronym{plic}{PLIC}{Platform-Level Interrupt Controller}
\newacronym{clint}{CLINT}{Core-local Interrupt Controller}
\newacronym{isa}{ISA}{Instruction set architecture}
\newacronym{machine}{M}{Machine}
\newacronym{user}{U}{User}
\newacronym{supervisor}{S}{Supervisor}
\newacronym{csr}{CSR}{Control and status register}
\newacronym{ooo}{OoO}{Out-of-Order}
\newacronym{os}{OS}{Operating System}
\newacronym{hdl}{HDL}{Hardware Description Language}
\newacronym{mmu}{MMU}{Memory Management Unit}
\newacronym{chisel}{Chisel}{Constructing Hardware in a Scala Embedded Language}
\newacronym{rtl}{RTL}{register-transfer level}
\newacronym{fpga}{FPGA}{Field-programmable gate array}
\newacronym{asic}{ASIC}{Application-Specific Integrated Circuit}