\chapter{Introduction}
\label{chapter:introduction}
This chapter discusses the motivation for developing this thesis project, the project's objectives and the results delivered. Furthermore, the chapter also covers hardware and software components designed to achieve the thesis goal. Lastly, the author explains the overall composition of this thesis report.

\section{Motivation}
\label{section:motivation}
The availability of fully open-source systems capable of executing an \acrfull{os} is limited. For a long time, the Linux kernel~\cite{torvalds1997linux} and the open-source software built around it allowed developers to implement a fully open-source Linux \acrshort{os} on their closed-source hardware devices. However, the lack of open-source hardware makes it difficult to develop fully open-source systems. With the appearance of \textit{RISC-V}~\cite{asanovic2014instruction}, open-source hardware availability started growing. Developing a \textit{RISC-V} \acrfull{soc} capable of running a Linux \acrshort{os} allows researchers to execute an \acrshort{os} in a fully open-source system. Having a Linux \acrshort{os} running in an \acrshort{soc} enables developers to create new applications for that \acrshort{soc} without worrying about its hardware components. The Linux community is significant, and researchers are used to working with the Linux kernel. Therefore, the requirement for an \acrshort{soc} capable of running Linux is high.

A Linux \acrshort{os} allows using many features unavailable in bare-metal applications. When developers create a bare-metal application, they are limited on software functionalities and must be aware of the \acrshort{soc} hardware characteristics. Similarly, if developers were to build an application using \acrfull{rtos}, for example, freeRTOS~\cite{barry2008freertos}, they would only have access to features such as a scheduler, events, threads, semaphores and message boxes. A Linux \acrshort{os} provides those and more functionalities. A Linux \acrshort{os} implements memory management and protection mechanisms, allows the execution of multiple applications simultaneously, supports various network adapters, and can interact with the user through a terminal. A Linux \acrshort{os} is also more secure than bare-metal or RTOS applications since it limits the user application's access to the machine resources, preventing misuse or damage.

The development of a \textit{RISC-V} \acrshort{soc} capable of running a Linux \acrshort{os} allows future open-source developments. Such as producing hardware accelerators which work with a Linux \acrshort{os} and integrating them with \textit{IOb-SoC-Linux}. These, and the possibilities to test in a real-world application, were the main reasons and motivations for developing this thesis.

\section{Objectives and Deliverables}
\label{section:objectives}
This study aims to develop an open-source \acrshort{soc} and execute a minimal Linux \acrshort{os}. The author will adapt the existing \textit{IOb-SoC}~\cite{iob_soc} to make an \acrshort{soc} that supports a Linux \acrshort{os}. \textit{IOb-SoC} is a modular open-source \textit{RISC-V} \acrshort{soc} that allows researchers to develop their own \acrshort{soc}. The \textit{IObundle} developers use \textit{Verilog}~\cite{thomas2008verilog} to describe \textit{IOb-SoC} and peripherals hardware.

The author had first to swap the \textit{IOb-SoC} \acrshort{cpu}. The problem with the current \acrshort{cpu} is that it cannot run an \acrshort{os}, only bare-metal applications. Therefore, \textit{IOb-SoC-Linux} contains a 32-bit \textit{RISC-V} \acrshort{cpu} capable of running Linux. Then, since the \textit{IOb-SoC} does not support interruption, the author had to create and integrate into \textit{IOb-SoC-Linux} the hardware needed to generate interrupts in a \textit{RISC-V} \acrshort{soc}. Lastly, the author had to ensure that the Linux \acrshort{os} supports the \acrshort{soc} \acrshort{uart}. Since Linux does not support the \textit{IOb-SoC} \acrshort{uart}, the author integrated a \textit{UART16550} in \textit{IOb-SoC-Linux}.

Four major software components make up a Linux \acrshort{os}. Those software components are the Linux kernel, the bootloader firmware, the \acrfull{rootfs} and a \acrfull{dtb}. The author built those software components to run a Linux \acrshort{os} on the \textit{IOb-SoC-Linux}. On power-on, the \textit{IOb-SoC-Linux} transfers the Linux \acrshort{os} software binary files onto the board where it runs, and the Linux \acrshort{os} will boot. After the \acrshort{os} boots, the user can run custom applications and take advantage of the Linux \acrshort{os}. The author also automated and documented the process of generating and deploying the Linux \acrshort{os} to \textit{IOb-SoC-Linux}. So, after this work, creating new \acrshort{os}s with different characteristics will be straightforward.

Finally, the system was verified both on simulation and running on an \acrshort{fpga} board. The \textit{IOb-SoC} needed a fast \textit{Verilog} simulator to verify the Linux OS execution. Therefore, the author developed a simulation testbench using the free-of-charge and open-source \textit{Verilator}~\cite{snyder2010verilator} simulator.

\section{Author's Work}
\label{section:authors_work}
The author had to develop four hardware modules to build a \acrfull{soc} capable of executing a Linux \acrshort{os}. Those hardware modules allowed the integration of a new \acrshort{cpu}, a new \acrshort{uart} and the hardware needed to support interrupts in the \textit{IOb-SoC}. Besides integrating new hardware in the \textit{IOb-SoC}, the author made minor changes to the \textit{IOb-SoC} core. The newly used \acrshort{cpu} core was generated based on the \textit{SpinalHDL}~\cite{papon2017spinalhdl} \textit{VexRiscv}~\cite{vexriscv} platform. The \textit{VexRiscv} platform allowed the author to develop a \textit{VexRiscv} \acrshort{cpu} core that meets the requirements to execute an \acrlong{os}. Furthermore, the author still had to create a \acrshort{cpu} wrapper to adapt the \textit{VexRiscv} \acrshort{cpu} to the \textit{IOb-SoC}. Another hardware component the author had to develop was a wrapper for a \acrshort{uart} compatible with Linux. The author adapted the wishbone interface of an industry-standard \textit{UART16550} to the \textit{IOb-SoC}. Additionally, the author fully developed the \acrshort{clint} hardware. The \acrshort{clint} is a hardware component that generates timer and software-related interrupts for a \textit{RISC-V} system. Another hardware component which manages interrupts generated by other peripherals in a \textit{RISC-V} system is the \acrshort{plic}. The author adapted an existing \acrshort{plic} module to the system by creating an interface that worked with \textit{IOb-SoC}.

During this thesis, the author also developed many software components. Those software components were essential to run a Linux \acrshort{os} in \textit{IOb-SoC} or enhance the \textit{IOb-SoC} platform. First, the author translated a program written in \textit{C} , called \textit{Console}, to \textit{Python}. The \textit{IOb-SoC} platform uses the \textit{Console} program to communicate through Serial with the board. Furthermore, the author added features to the \textit{Console} that made it capable of working with the simulator testbench and communicating with a Linux \acrshort{os} running in \textit{IOb-SoC}. Secondly, the author developed a new simulation testbench, based on the previous \textit{IOb-SoC} verification software, capable of communicating with the \textit{Console} program. Moreover, the author integrated the \textit{Verilator}~\cite{snyder2010verilator} simulation software in \textit{IOb-SoC} and created a \textit{Verilator} \textit{C++} testbench. Thirdly, the author created a simulation testbench for the \acrshort{clint} hardware and an interrupt routine firmware that took advantage of the \acrshort{clint}. The firmware created demonstrates how to use interrupts in \textit{IOb-SoC}. Finally, the author adapted, built and deployed the software needed to execute a Linux \acrshort{os} in the \acrshort{soc}. The author had to adjust the \textit{IOb-SoC} bootloader firmware. He created a device tree file describing the hardware components of the \acrshort{soc}. The author compiled a Linux kernel version compatible with the \textit{VexRiscv} \acrshort{cpu} and developed a \acrlong{rootfs} adequate for a minimal Linux \acrshort{os}. While developing the hardware and software components, the author also developed Makefile scripts that help integrate the components in \textit{IOb-SoC} and automatise the building and deployment process.

\section{Thesis Outline}
\label{section:thesis_outline}
This thesis has five significant chapters. Chapters \ref{chapter:must_have_concepts} and \ref{chapter:existing_embedded_technologies} address the state-of-the-art. Chapters \ref{chapter:hardware_developed} and \ref{chapter:software_developed} discuss the authors' developments to achieve the thesis goal, and chapter \ref{chapter:project_results} shows the results. 

The first chapter introduces the thesis, and the last chapter (\ref{chapter:conclusions}) is the author's conclusions after completing the project. Chapter \ref{chapter:must_have_concepts} discusses the tools, concepts and standards the reader must understand to comprehend the following chapters and the author's decisions while developing the project. Chapter \ref{chapter:existing_embedded_technologies} presents the existing technologies capable of executing a full-feature \acrshort{os} and discusses the possibility of integrating them in \textit{IOb-SoC}. The author described the hardware he had to develop to successfully create an \acrshort{soc} capable of executing a Linux \acrshort{os} in chapter \ref{chapter:hardware_developed}. Chapter \ref{chapter:software_developed} presents the software components the author had to make. The software is required to run the \acrshort{os} in the developed \acrshort{soc}, improve the \textit{IOb-SoC} platform, and test the hardware the author created. Finally, chapter \ref{chapter:project_results} shows the products of this expedition. The author achieved many breakthroughs in various steps of this thesis development. For every breakthrough, the author obtained results that allowed analysing and discussing the hardware and software developed up until then.
