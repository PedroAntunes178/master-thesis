\cleardoubleoddpage

\chapter*{Abstract}
\thispagestyle{empty} %hide page numbers
% up to 250 words
% Summarize the problem
The recent appearance of the RISC-V ISA opened many exciting possibilities for building processor-based systems without the need to license the base architecture from providers like ARM. Running applications on bare metal RISC-V machines is a good starting point, but an OS is required to ease the developers' efforts for more complex applications. Linux has been around for over three decades and is a well-polished OS. The problem is that open-source SoC platforms that run Linux and, at the same time, are modular and configurable do not exist.
% Solution found
This work aims to implement Linux on IOb-SoC, a modular and configurable open-source SoC platform that only runs bare-metal applications. A suitable 32-bit RISC-V CPU capable of running Linux on the Iob-SoC will be adopted or developed. A process for generating, booting and running configurable Linux images will be created. Emulation, HDL simulation, and FPGA prototyping will thoroughly verify this hardware/software system.


\vfill

\textbf{\Large Keywords:} RISC-V, Linux, Systems on-Chip (SoC), Verilog, IOb-SoC