\cleardoubleoddpage

\chapter*{Abstract}
\thispagestyle{empty} %hide page numbers
% up to 250 words
% Summarize the problem
The recent appearance of the \textit{RISC-V} \acrshort{isa} opened many exciting possibilities for building processor-based systems without the need to license the base architecture from providers like ARM. Running applications on bare metal \textit{RISC-V} machines is a good starting point, but an \acrshort{os} is required to ease the developers' efforts for more complex applications. Linux has been around for over three decades and is a well-polished \acrshort{os}. The problem is that open-source \acrshort{soc} platforms that run Linux and simultaneously are modular and configurable do not exist.
% Solution found
This work aims to create a \acrshort{soc} capable of executing a Linux \acrshort{os}. The author based the work on \textit{IOb-SoC}, a modular and configurable open-source \acrshort{soc} platform that only runs bare-metal applications. The author achieved this work's goal by changing the \textit{IOb-SoC} \acrshort{cpu} and adding three hardware peripherals. Additionally, the author developed software that improved the \textit{IOb-SoC} platform, complemented the hardware components created and allowed the execution of a complete \acrshort{os} in the new \acrshort{soc}. Along this work, the \acrshort{soc} developed might also be referred as \textit{IOb-SoC-Linux}.
% Results
The \textit{IOb-SoC-Linux} uses less than 10\% of the \acrshort{fpga} resources on the supported development boards. Moreover, the Linux \acrshort{os} boots in five seconds in the Kintex Ultrascale and seven seconds in the Cyclone V.


\vfill

\textbf{\Large Keywords:} RISC-V, Linux, Systems on-Chip (SoC), Verilog, IOb-SoC