\cleardoubleoddpage

\chapter*{Abstract}
\thispagestyle{empty} %hide page numbers
% up to 250 words
% Summarize the problem
The recent appearance of the \textit{RISC-V} \acrshort{isa} opened many exciting possibilities for building processor-based systems without the need to license the base architecture from providers like Arm Holdings (Arm ®). Running applications on bare metal \textit{RISC-V} machines is a good starting point, but an \acrshort{os} is required to ease the developers' efforts for more complex applications. Linux is a well-polished \acrshort{os} since it has been used for over three decades. The problem is that open-source \acrshort{soc} platform solutions that run Linux and simultaneously are modular and configurable do not exist.
% Solution found
This work aims to create an \acrshort{soc} capable of executing a Linux \acrshort{os}. The author based the work on \textit{IOb-SoC}, a modular and configurable open-source \acrshort{soc} platform that only runs bare-metal applications. This project achieves its goals by changing the \textit{IOb-SoC} \acrshort{cpu} and adding three hardware peripherals. Additionally, the author develops software solutions that improve the \textit{IOb-SoC} platform, complement the hardware components created and enhance the hardware to allow the execution of a complete \acrshort{os} in a new \acrshort{soc} called \textit{IOb-SoC-Linux}.
% Results
The size of \textit{IOb-SoC-Linux} is only marginally above that of the original \textit{IOb-SoC} and can run in most low-cost \acrshort{fpga}s. The Linux \acrshort{os} takes four minutes and thirty seconds to build. Linux boots in a \textit{Kintex Ultrascale} device in five seconds and in seven seconds in a \textit{Cyclone V} device. The work developed in this thesis met all the project's goals and went beyond them.


\vfill

\textbf{\Large Keywords:} RISC-V, Linux, Systems on-Chip (SoC), Verilog, IOb-SoC