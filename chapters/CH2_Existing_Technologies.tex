\chapter{Existing Technologies}
There already exists embedded microcontrollers capable of running Linux. Big companies as for example ARM, Qualcomm, MediaTek, Intel and AMD have created microcontroller capable of running Linux. But the processor architecture of those microcontrollers is not open-source, much less the microcontroller itself. 

As an example, the \textit{Raspberry Pi 4} is a very capable and cheap board where a developer can test and implement new software running in Linux. The Raspberry CPU is an \textit{Cortex-A72}~\cite{cortex_a72} witch is a System on Chip (SoC) developed by ARM on their ARMv8 64-bit CPU architecture. But if someone wanted to use the Raspberry as a base for his costume hardware design, that would be impossible. And thus appears the need for open-source hardware that allows creating something new without having to start from scratch every time. This led to the appearance of RISC-V the open-source CPU architecture.


\section{Close source RISC-V Embedded Systems}
Since then, a few companies using RISC-V have appeared. RISC-V CPUs are already present in the automotive and IoT markets, besides AI chips in data centers. Due to the RISC-V ISA royalty free license new StartUps tend to look at RISC-V CPUs as a solution for their cores. Even if the CPU Core isn't free to use it ends up being a cheaper solution.

While creating new products companies proved how advantageous the RISC-V architecture was. Furthermore, they have contributed to open-source software, hardware and documentation. Some companies with a big recognitions involved with RISC-V technology are:
\begin{itemize}
    \item \textit{Western Digital} who now uses RISC-V in its external storage disks. 
    \item \textit{Microchip} as launched the first RISC-V-Based System-on-Chip (SoC) FPGA, \textit{PolarFire}. 
    \item \textit{Antmicro/Microsemi}~\footnote{Microchip has acquired Microsemi Corporation in May 2018.} have built a software called Renode that is used to develop, debug and test multi-node RISC-V device systems.
    \item \textit{BeagleBoard.org}, \textit{Seeed Studio} and \textit{StarFive} worked together to build the first affordable RISC-V computer designed to run Linux, \textit{BeagleV}~\cite{beagleV}. The board is priced around 150€.
\end{itemize}

These companies have all helped pave the way for a full-feature Operating System based on the Linux kernel to be compatible with the RISC-V architecture. However, there are two companies that have a bigger impact on RISC-V CPU design, those are Andes Technology and SiFive.

\subsection{Andes Technology}
Andes Technology is one of the founder members of the RISC-V International. Since it is highly involved with RISC-V it ended up being one of the major contributor (and maintainer) of the RISC-V tool-chain. This is important because the RISC-V ISA is merely an instruction set architecture, there needs to exist complementing software, such as compiler and development tools.

Nowadays Andes CPU's are applied nearly everywhere, from telecommunications, storage controllers, touch screen sensors to data centers, etc. Andes Technologies has had incredible success using RISC-V technology, as prove they have shipped billions of embedded SoC with RISC-V processors based on their RISC-V ISA variant, AndeStar™ V5. 

Andes CPUs witch are capable of running Linux are the \textit{A25}~\cite{a25} and \textit{AX25}~\cite{ax25}. Both support single and double precision floating point, the RISC-V P-extension (draft) DSP/SIMD ISA and an MMU (Memory Management Unit) for Linux applications. Besides that both enable the use of Machine (M), User (U) and Supervisor (S) Privilege levels that allow running Linux and other advanced operating systems with protection between kernel and user programs. Furthermore, both have L1 instruction and data cache. The difference between them is that \textit{A25} is based on 32-bit architecture and the \textit{AX25} is 64-bit. This leads to the \textit{AX25} being ideal for embedded applications that need to access address space over 4GB, and the \textit{A25} being smaller in gate count. Both CPUs can be implemented on the AE350~\cite{ae350} SoC allowing to use these CPUs on developer boards, for example in the \textit{ADP-XC7K160/410}~\cite{adp-xc7k160}.

\subsection{SiFive}
SiFive is a company that was born from the RISC-V ISA. SiFive was founded by three researchers from the University of California Berkeley, Krste Asanović, Yunsup Lee, and Andrew Waterman. Those researchers were deeply involved with the development of the RISC-V ISA, from working on the base ISA to working on the floating point numbers and compressed instructions ISA extensions. It is no surprise that the first company to release a chip and development board that implemented the RISC-V ISA was SiFive. This happened in 2016 one year after the company was founded.

In 2017 SiFive launched \textit{U54}~\cite{u54} witch was the first RISC-V CPU capable of running a full fledge Operating System like Linux. With it they launched the \textit{U54-MC}~\cite{u54-mc} SoC that had four \textit{U54} 64-bit cores. Furthermore, the \textit{U54-MC} implemented the initial CLINT and PLIC unit. The development of the CLINT and the PLIC made by SiFive would eventually lead to the documentation and specification of the respective hardware components with witch RISC-V systems must be compliant if they proclaim to use either one. One year after, in 2018, they launched \textit{HiFive Unleashed}~\cite{hifive_unleashed} witch was the first board that implemented the \textit{U54} CPU and run a Linux OS with a desktop environment (DE). Since then the \textit{HiFive Unleashed} has been discontinued and better hardware has been made available.

\textit{HiFive Unmatched}~\cite{hifive_unmatched} / \textit{U74-MC}~\cite{u74-mc} SoC

\section{Open Source Solutions}
Built upon the RISC-V open-source CPU architecture, various CPU designs have emerged. RISC-V CPUs are most popular in ... . Some well known CPU are ... . Those will not be discussed here sice they do not meet the requirements to run the Linux Kernel. To run a Linux based Operating System, for example, the ... would have to ... .

\subsection{PULP Platform}
\textit{CVA6}~\cite{zaruba2019cost} \textit{OpenPiton}~\cite{Balkind:2016:OOS:2872362.2872414}

\subsection{The Berkeley Out-of-Order RISC-V Processor}
The Berkeley Out-of-Order RISC-V Processor (\textit{BOOM}~\cite{zhaosonicboom}) ... 
rocket-chip from CHIPS Alliance

\subsection{SpinalHDL}
VexRiscv and NaxRiscv
Talk about litex-vexriscv!


\begin{table}[h]
    \centering
    \resizebox{\textwidth}{!}{%
    \begin{tabular}{ccccclcccc}
                                       & ARM                             & \multicolumn{2}{c}{Andes Technology}                      & \multicolumn{2}{c}{SiFive}                                & PULP plataform              & UC Berkeley                 & \multicolumn{2}{c}{SpinalHDL}                                 \\ \cline{2-10} 
    \multicolumn{1}{c|}{}              & \multicolumn{1}{c|}{Cortex-A72} & \multicolumn{1}{c|}{A25}    & \multicolumn{1}{c|}{AX25}   & \multicolumn{1}{c|}{U54}    & \multicolumn{1}{l|}{U74}    & \multicolumn{1}{c|}{CVA6}   & \multicolumn{1}{c|}{BOOM}   & \multicolumn{1}{c|}{VexRiscv} & \multicolumn{1}{c|}{NaxRiscv} \\ \hline
    \multicolumn{1}{|c|}{Architecture} & \multicolumn{1}{c|}{64-bit}     & \multicolumn{1}{c|}{32-bit} & \multicolumn{1}{c|}{64-bit} & \multicolumn{1}{c|}{64-bit} & \multicolumn{1}{l|}{64-bit} & \multicolumn{1}{c|}{64-bit} & \multicolumn{1}{c|}{64-bit} & \multicolumn{1}{c|}{32-bit}   & \multicolumn{1}{c|}{64-bit}   \\ \hline
    \multicolumn{1}{|c|}{}             & \multicolumn{1}{c|}{}           & \multicolumn{1}{c|}{}       & \multicolumn{1}{c|}{}       & \multicolumn{1}{c|}{}       & \multicolumn{1}{l|}{}       & \multicolumn{1}{c|}{}       & \multicolumn{1}{c|}{}       & \multicolumn{1}{c|}{}         & \multicolumn{1}{c|}{}         \\ \hline
    \multicolumn{1}{|c|}{Open-Source}  & \multicolumn{1}{c|}{N}          & \multicolumn{1}{c|}{N}      & \multicolumn{1}{c|}{N}      & \multicolumn{1}{c|}{N}      & \multicolumn{1}{l|}{N}      & \multicolumn{1}{c|}{Y}      & \multicolumn{1}{c|}{Y}      & \multicolumn{1}{c|}{Y}        & \multicolumn{1}{c|}{Y}        \\ \hline
    \end{tabular}%
    }
    \caption{CPU comparation table}
    \label{tab:cpu_comparation}
\end{table}