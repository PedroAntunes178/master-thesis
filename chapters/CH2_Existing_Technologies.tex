\chapter{Existing Technologies}
There already exists embedded microcontrollers capable of running Linux. Big companies as for example ARM, Qualcomm, MediaTek, Intel and AMD have created microcontroller capable of running Linux. But the processor architecture of those microcontrollers is not open-source, much less the microcontroller itself. 

As an example, the \textit{Raspberry Pi 4} is a very capable and cheap board where a developer can test and implement new software running in Linux. The Raspberry CPU is an \textit{Cortex-A72}~\cite{cortex_a72} witch is a System on Chip (SoC) developed by ARM on their ARMv8 64-bit CPU architecture. But if someone wanted to use the Raspberry as a base for his costume hardware design, that would be impossible. And thus appears the need for open-source hardware that allows creating something new without having to start from scratch every time. This led to the appearance of RISC-V the open-source CPU architecture.

\section{Close source RISC-V Microcontrollers}
Since then, a few companies using RISC-V have appeared. For example, \textit{Western Digital} now uses RISC-V in its external storage disks. \textit{Microchip} as launched the first RISC-V-Based System-on-Chip (SoC) FPGA, \textit{PolarFire}. 

These companies have helped pave the way for the Linux kernel to be compatible with the RISC-V architecture. But there are two companies that have contributed and boosted the RISC-V technology above all those are Andes Technology and Sifive .

\subsection{Andes Technology}
Andes Technology is one of the founder members of the RISC-V International S. Due to its involvement with RISC-V it is one of the major contributors to the RISC-V toolchain that all RISC-V developers use.  Andes Technologies have shipped billions of embedded SoC with RISC-V processors based on AndeStar™ V5 architecture.

Nowadays Andes CPU's are applied nearly everywhere, for example in telecommunications, storage controllers, data centers, touch screen sensors, etc. Some of their products that where captivating are 

\subsection{Sifive}


\section{Open Source Solutions}
Built upon the RISC-V open-source CPU architecture, various CPU designs have emerged. RISC-V CPUs are most popular in ... . Some well known CPU are ... . Those will not be discussed here sice they do not meet the requirements to run the Linux Kernel. To run a Linux based Operating System, for example, the ... would have to ... .

\subsection{CVA6/OpenPiton}
\subsection{RocketChip}
\subsection{NaxRiscv}
\subsection{VexRiscv}

Talk about litex-vexriscv!