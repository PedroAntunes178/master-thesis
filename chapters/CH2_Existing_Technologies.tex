\chapter{Existing Technologies}
There already exists embedded microcontrollers capable of running Linux. Big companies as for example ARM, Qualcomm, MediaTek, Intel and AMD have created microcontroller capable of running Linux. But the processor architecture of those microcontrollers is not open-source, much less the microcontroller itself. 

As an example, the \textit{Raspberry Pi 4} is a very capable and cheap board where a developer can test and implement new software running in Linux. The Raspberry CPU is an \textit{Cortex-A72}~\cite{cortex_a72} witch is a System on Chip (SoC) developed by ARM on their ARMv8 64-bit CPU architecture. But if someone wanted to use the Raspberry as a base for his costume hardware design, that would be impossible. And thus appears the need for open-source hardware that allows creating something new without having to start from scratch every time. This led to the appearance of RISC-V the open-source CPU architecture.


\section{Close source RISC-V Embedded Systems}
Since then, a few companies using RISC-V have appeared. RISC-V CPUs are already present in the automotive and IoT markets, besides AI chips in data centers. Due to the RISC-V ISA royalty free license new StartUps tend to look at RISC-V CPUs as a solution for their cores. Even if the CPU Core isn't free to use it ends up being a cheaper solution.

While creating new products companies proved how advantageous the RISC-V architecture was. Furthermore, they have contributed to open-source software, hardware and documentation. Some companies with a big recognitions involved with RISC-V technology are:

\begin{itemize}
    \item \textit{Western Digital} who now uses RISC-V in its external storage disks. 
    \item \textit{Microchip} as launched the first RISC-V-Based System-on-Chip (SoC) FPGA, \textit{PolarFire}. 
    \item \textit{Antmicro/Microsemi}~\footnote{Microchip has acquired Microsemi Corporation in May 2018.} have built a software called Renode that is used to develop, debug and test multi-node RISC-V device systems.
    \item more???
\end{itemize}

These companies have all helped pave the way for a full-feature Operating System based on the Linux kernel to be compatible with the RISC-V architecture. However, there are two companies that have a bigger impact on RISC-V CPU design, those are Andes Technology and Sifive.

\subsection{Andes Technology}
Andes Technology is one of the founder members of the RISC-V International. Due to its involvement with RISC-V it is one of the major contributor (and maintainer) to the RISC-V tool-chain. The RISC-V ISA is merely an instruction set architecture, there needs to be development in complementing tools, such as compiler and development tools, that all RISC-V developers use.

Nowadays Andes CPU's are applied nearly everywhere, for example in telecommunications, storage controllers, data centers, touch screen sensors, etc. Andes Technologies has shipped billions of embedded SoC with RISC-V processors based on AndeStar™ V5 architecture. 

Besides contributing to the tool-chain, since Andes also developed CPUs capable of running Linux they have presented numerous talks about

\subsection{Sifive}
Sifive Founder, ..., was one of the bercley students who worked on RISC-V.
In 2017 Sifive launched \textit{'U54-MC'} witch was the first RISC-V CPU capable of running a full fledge Operating System like Linux. The SoC had a 64-bit architecture and was quad-core.
PLIC, CLINT, MMU


\section{Open Source Solutions}
Built upon the RISC-V open-source CPU architecture, various CPU designs have emerged. RISC-V CPUs are most popular in ... . Some well known CPU are ... . Those will not be discussed here sice they do not meet the requirements to run the Linux Kernel. To run a Linux based Operating System, for example, the ... would have to ... .

\subsection{PULP Platform}
CVA6/OpenPiton
\subsection{CHIPS Alliance}
rocket-chip
\subsection{SpinalHDL}
VexRiscv and NaxRiscv
Talk about litex-vexriscv!