\chapter{Hardware Developed}
During the development of this thesis there was both hardware and software developed. In this chapter we are going to go through the hardware developed in order to build an appropriate \acrfull{soc} capable of running a full-fledged \acrfull{os}.

The \textit{IOb-SoC} was used as a \acrfull{soc} template. \textit{IOb-SoC} has some features that make it ideal to develop this project \acrshort{soc}. Firstly, it is open-source hardware. Witch means there are no royalties and the source code is publicly available. Secondly, adding new peripherals is very easy and intuitive, as it was previously seen in section \ref*{section:the_iob_soc_template}. Finally, it already has some features that make it ideal to use. \textit{IOb-SoC} already implements the interface with an internal (SRAM) and an external (DRAM) memory, contains iob-cache system and boot hardware unit. The boot hardware unit controls the first boot stage (also known as stage zero) that is executed after powering/resetting the system.

The hardware components that needed to be changed from \textit{IOb-SoC} were the \acrfull{cpu} and the \acrfull{uart} peripheral. The \acrshort{cpu} had to be changed because the previous \acrshort{cpu} (\textit{PicoRV32}) is not capable of running a full-feature \acrlong{os}. The \acrshort{uart} had to be swapped since there were no compatible Linux drivers that worked with \textit{iob-UART}. Besides swapping a few components from the chip new hardware had to be added. The additional hardware is the \acrshort{clint} and the \acrshort{plic} both compatible with RISC-V specifications. The \acrshort{clint} was added to implement timer and software interrupts on the \acrshort{soc}. The \acrshort{plic} was added to manage interrupts generated by other peripherals, for example from the \acrshort{uart}. A sketch of the \acrshort{soc} developed can be seen in figure \ref{fig:bd_linux}.

\begin{figure}[!h]
    \centering
    \includegraphics[width=0.7\linewidth]{bd_linux.pdf}
    \caption{Developed \acrshort{soc} sketch.}
    \label{fig:bd_linux}
\end{figure}

Comparing figure \ref{fig:bd_linux} with the original design of \textit{IOb-SoC} (figure \ref{fig:bd_original}) we can see that there were a few additional alterations. In the first place, it can be seen that the L1 Cache was removed. Since every application processor studied had a L1 cache built in, there was no need for the L1 \textit{iob-cache}. Next, a \textit{iob-split} was added to the \textit{IOb-SoC}. Previously, there was a single \textit{iob-split} for the data bus with three slaves (the internal memory, the external memory and the peripheral bus). This meant that there were 2 selection bits, when '00' then the internal memory bus was active, when '01' it was the peripheral bus and when '10' it was the external memory. This caused a problem because when addressing the external memory if its size is bigger than 1GB the selection bits would be '11'. The \acrfull{demux} output selected by '11' is not connected to anywhere, so this caused an internal hardware error. The solution was to include two \textit{iob-split} modules each one with two slaves. The first would chose between the external memory and either the internal memory or peripheral bus. The second would chose between the internal memory and the peripheral bus. Another advantage of using this method is that now the selection bits position does not vary depending on if we are using the DDR or not. This makes it easier to use external software that does not make use of the \textit{iob-soc} Makefiles. Before the peripheral addressing on external software had to be changed every time the developer wanted to test with or without the external memory.

During this project there was also an improvement on the \textit{IOb-SoC} verification. This lead to the creation of a top hardware module for the developed \acrlong{soc}.

\section{Central Processing Unit}
The \acrshort{cpu} chosen to use in this project was \textit{VexRiscv}. \textit{VexRiscv} could be configured in various ways
\subsection{VexRiscv Wrapper}

\section{UART16550 Wrapper}
The approach taken in this project was to adapt an existing open-source UART core that is supported by the Linux kernel. The other option was to create a Linux driver compatible with \textit{iob-UART} and compile the kernel with it. The chosen approach seamed more adequate and simpler solution.

\section{CLINT Unit}
\acrshort{clint}

\section{PLIC Unit Wrapper}
\acrshort{plic} 

\section{UUT Top Hardware}
This top module creates a verilog wrapper of the \acrfull{uut} that allows it to interact with the different hardware logic simulators. The top module file is an adaptation of the previous verilog file used on icarus simulation.

\begin{figure}[!ht]
    \centering
    \includegraphics[width=\linewidth]{uut_top_hw.pdf}
    \caption{Simulated hardware interfaces.}
    \label{fig:uut_top_hw}
\end{figure}