\cleardoubleoddpage

\chapter*{Resumo}

O aparecimento da arquitetura de computadores \textit{RISC-V} levou ao desenvolvimento de “hardware” totalmente “open-source”. Para trabalhar com esse “hardware” não são necessárias licenças de fornecedores como a Arm Holdings (Arm ®). O desenvolvimento de aplicações “bare-metal” para sistemas \textit{RISC-V} é um bom ponto de partida. No entanto, a integração de um SO Linux facilita aos desenvolvedores criar  aplicações e utilizar “software” já existente. O problema é que não existem “SoC” de código aberto que consigam correr o Linux e, simultaneamente, ser modulares.  
%  
Com este trabalho pretende-se criar um “SoC”, baseado no \textit{IOb-SoC}, capaz de correr o Linux. O \textit{IOb-SoC} é um “SoC” modular e configurável que só funciona com aplicações “bare-metal”. O “SoC” desenvolvido chama-se \textit{IOb-SoC-Linux}. Os resultados obtidos mostram que recursos utilizados pelo \textit{IOb-SoC-Linux} são poucos mais que pelo \textit{IOb-SoC} de modo que o \textit{IOb-SoC-Linux} consegue correr em praticamente qualquer FPGA de baixo custo. O SO Linux leva quatro minutos e trinta segundos a compilar. O “kernel” Linux leva cinco segundos a iniciar na placa \textit{Kintex Ultrascale} e sete na \textit{Cyclone V}. Os resultados demostram que os recursos utilizados pelo \textit{IOb-SoC-Linux} na \textit{Kintex Ultrascale} e na \textit{Cyclone V} são menos de 10\% dos recursos disponíveis nas FPGAs dessas placas.
%
Os objetivos do projeto foram atingidos após trocar o CPU do \textit{IOb-SoC} e adicionar componentes de “hardware” que suportam interrupções e os “drivers” de Linux para a UART. Foram também desenvolvidos programas que melhoram a plataforma do \textit{IOb-SoC}, complementam o “hardware” desenvolvido e permitem a execução do SO no \textit{IOb-SoC-Linux}.

\vfill

\textbf{\Large Palavras-chave:} RISC-V, Linux, Sistema-em-um-Chip (SoC), Verilog, IOb-SoC

