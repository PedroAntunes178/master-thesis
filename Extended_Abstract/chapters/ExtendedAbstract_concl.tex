%%%%%%%%%%%%%%%%%%%%%%%%%%%%%%%%%%%%%%%%%%%%%%%%%%%%%%%%%%%%%%%%%%%%%%
%     File: ExtendedAbstract_concl.tex                               %
%     Tex Master: ExtendedAbstract.tex                               %
%                                                                    %
%     Author: Andre Calado Marta                                     %
%     Last modified : 27 Dez 2011                                    %
%%%%%%%%%%%%%%%%%%%%%%%%%%%%%%%%%%%%%%%%%%%%%%%%%%%%%%%%%%%%%%%%%%%%%%
% The main conclusions of the study presented in short form.
%%%%%%%%%%%%%%%%%%%%%%%%%%%%%%%%%%%%%%%%%%%%%%%%%%%%%%%%%%%%%%%%%%%%%%

\section{Conclusions}
\label{sec:concl}

\subsection{Achievements}
The work developed in this thesis project successfully achieves the project's goals. The \textit{Verilator} simulation testbench created is much faster than the previous verification process, and using \textit{Verilator} saves time when verifying an SoC based on the IOb-SoC. Furthermore, the Python Console program developed works correctly with the simulation testbench and the FPGA boards.

The author successfully integrated a CPU that meets the requirements to run an OS and verified that what worked with the previous CPU still worked in the new SoC. The CPU integrated is the \textit{VexRiscv} CPU generated using the \textit{SpinalHDL}  \textit{VexRiscv} platform. Additionally, the author successfully created the CLINT component for timer and software interrupts, and the simulation testbench developed for the CLINT shows it works as expected. Moreover, the interrupt routine firmware developed, which takes advantage of the CLINT, shows how interrupts work in bare-metal with the \textit{IOb-SoC-Linux}. The PLIC integrated into \textit{IOb-SoC-Linux} allows the SoC to support interrupts from its peripheral hardware components. Furthermore, since the Linux OS does not support the \textit{IOb-SoC} current UART, in this thesis, the author adapts an industry-standard UART16550 to the \textit{IOb-SoC-Linux}. The number of resources the complete \textit{IOb-SoC-Linux} uses is less than 10\% of the supported FPGA boards. Comparing the \textit{IOb-SoC} resource consumption with the resources used by the \textit{IOb-SoC-Linux}, which can execute a Linux OS, the author can conclude that the developed SoC uses only a few more resources than the original. There is still plenty of space in the FPGA o implement new hardware accelerators.

The minimal Linux OS developed executes on both supported FPGA boards and simulation with the \textit{Verilator} testbench. The OpenSBI bootloader, the Device Tree Blob, the Linux kernel and the root file system constitute the Linux OS. The OpenSBI bootloader implements the \textit{RISC-V} SBI functions, which the supervisor mode software uses to communicate with the machine privilege level. The Device Tree Blob describes the \textit{IOb-SoC-Linux} hardware, which the Linux Kernel uses to know what drivers to use. The Linux kernel implements the system calls that the user applications can use. Lastly, the root file system uses the Busybox software package and allows users to interact with the Linux OS. The minimal Linux OS developed takes five seconds to boot in the \textit{Kintex Ultrascale} board and seven seconds in the \textit{Cyclone V}.

Finally, the Makefiles written in this thesis allow researchers to use the developed components easily. Building a complete Linux OS with the created Makefiles takes the user four minutes and thirty seconds.

\subsection{Future Work}
After completing the thesis objectives, there is still room for new features and optimisation. The author or other developer/researcher can submit new features to optimise \textit{IOb-SoC-Linux}. The author is working on four optimisations. First, the candidate could optimise the performance of the SoC if the \textit{VexRiscv} CPU integrated into \textit{IOb-SoC-Linux} supported 32 bytes per cache line. Currently, the CPU has an L1 data and instructions cache with 4 bytes per line. Secondly, right now, \textit{IOb-SoC-Linux} does not have support for internet connections. Therefore, the author will adapt an existing Ethernet controller to the \textit{IOb-SoC-Linux} by creating a hardware wrapper. Thirdly, currently, \textit{IOb-SoC-Linux} has to transfer the Linux OS every time it starts working and transmitting data through the UART is slow. Integrating a Serial Peripheral Interface (SPI) controller would allow \textit{IOb-SoC-Linux} to load the software from flash memory. An alternative solution would be to implement a PCI interface and transfer the data through it. Lastly, the author will optimise the Console program. With the existing program, the user input is not fluid since the Console software does the input processing sequentially after the program waits a short period for data to be read from the serial connection. The optimised Console program should receive the user input and read from the serial interface concurrently in two different threads.

One of the best advantages of this thesis project is the opportunities it creates. Many possible projects could use \textit{IOb-SoC-Linux}. The author is currently involved in a project called OpenCryptoLinux, which the NLnet Foundation has funded through the NGI Assure Fund with financial support from the European Commission's Next Generation Internet programme. OpenCryptoLinux aims to adapt the OpenCryptoHW [19] project to \textit{IOb-SoC-Linux}. Therefore creating a secure and user-friendly open-source SoC template with cryptography functions running a Linux OS on a \textit{RISC-V} system. A colleague working at IObundle develops OpenCryptoHW, which implements a reconfigurable open-source cryptographic hardware IP core. The hardware is reconfigurable because the CPU controls low-cost Coarse-Grained Reconfigurable Arrays (CGRAS). OpenCryptoLinux can enhance the security, privacy, performance, and energy efficiency of future Internet of Things (IoT) devices. The OpenCryptoLinux project will be fully open-source, guaranteeing public scrutiny and quality. The author has to develop Linux drivers that can control the openCryptoHW hardware and possibly integrate a DMA controller in the \textit{IOb-SoC-Linux} to integrate openCryptoHW features in the Linux OS. Finally, it would also be interesting to implement the \textit{IOb-SoC-Linux} as an ASIC and create a development board with it at its core.
