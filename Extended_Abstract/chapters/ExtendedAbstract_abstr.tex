%%%%%%%%%%%%%%%%%%%%%%%%%%%%%%%%%%%%%%%%%%%%%%%%%%%%%%%%%%%%%%%%%%%%%%
%     File: ExtendedAbstract_abstr.tex                               %
%     Tex Master: ExtendedAbstract.tex                               %
%                                                                    %
%     Author: Andre Calado Marta                                     %
%     Last modified : 2 Dez 2011                                     %
%%%%%%%%%%%%%%%%%%%%%%%%%%%%%%%%%%%%%%%%%%%%%%%%%%%%%%%%%%%%%%%%%%%%%%
% The abstract of should have less than 500 words.
% The keywords should be typed here (three to five keywords).
%%%%%%%%%%%%%%%%%%%%%%%%%%%%%%%%%%%%%%%%%%%%%%%%%%%%%%%%%%%%%%%%%%%%%%

%%
%% Abstract
%%
\begin{abstract}

The recent appearance of the RISC-V ISA opened many exciting possibilities for building processor-based systems without the need to license the base architecture from providers like ARM. Running applications on bare metal RISC-V machines is a good starting point, but an OS is required to ease the developers' efforts for more complex applications. Linux has been around for over three decades and is a well-polished OS. The problem is that open-source SoC platforms that run Linux and simultaneously are modular and configurable do not exist. This work aims to create an SoC capable of executing a Linux OS. The author bases the work on IOb-SoC, a modular and configurable open-source SoC platform that only runs bare-metal applications. The author achieves this thesis goal by changing the IOb-SoC CPU and adding three hardware peripherals. Additionally, the author develops software that improves the IOb-SoC platform, complements the hardware components created and allows the execution of a complete OS in the new SoC. Throughout this work, the thesis might refer to the SoC developed as IOb-SoC-Linux. The IOb-SoC-Linux uses less than 10\% of the FPGA resources on the supported development boards. Moreover, the Linux OS boots in five seconds in the Kintex Ultrascale and seven seconds in the Cyclone V.
\\
%%
%% Keywords (max 5)
%%
\noindent{{\bf Keywords:}} RISC-V, Linux, Systems on-Chip (SoC), Verilog, IOb-SoC \\

\end{abstract}

