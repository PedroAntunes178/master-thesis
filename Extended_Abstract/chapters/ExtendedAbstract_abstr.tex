%%%%%%%%%%%%%%%%%%%%%%%%%%%%%%%%%%%%%%%%%%%%%%%%%%%%%%%%%%%%%%%%%%%%%%
%     File: ExtendedAbstract_abstr.tex                               %
%     Tex Master: ExtendedAbstract.tex                               %
%                                                                    %
%     Author: Andre Calado Marta                                     %
%     Last modified : 2 Dez 2011                                     %
%%%%%%%%%%%%%%%%%%%%%%%%%%%%%%%%%%%%%%%%%%%%%%%%%%%%%%%%%%%%%%%%%%%%%%
% The abstract of should have less than 500 words.
% The keywords should be typed here (three to five keywords).
%%%%%%%%%%%%%%%%%%%%%%%%%%%%%%%%%%%%%%%%%%%%%%%%%%%%%%%%%%%%%%%%%%%%%%

%%
%% Abstract
%%
\begin{abstract}

The recent appearance of the \textit{RISC-V} ISA opened many exciting possibilities for building processor-based systems without the need to license the base architecture from providers like Arm Holdings (Arm ®). Running applications on bare metal \textit{RISC-V} machines is a good starting point, but an OS is required to ease the developers' efforts for more complex applications. Linux is a well-polished OS since people have been using it for over three decades. The problem is that open-source SoC platform solutions that run Linux and simultaneously are modular and configurable do not exist. This work aims to create an SoC capable of executing a Linux OS. The author based the work on \textit{IOb-SoC}, a modular and configurable open-source SoC platform that only runs bare-metal applications. This project achieves its goals by changing the \textit{IOb-SoC} CPU and adding three hardware peripherals. Additionally, the author develops software solutions that improve the \textit{IOb-SoC} platform, complement the hardware components created and enhance the hardware to allow the execution of a complete OS in a new SoC called \textit{IOb-SoC-Linux}. The size of \textit{IOb-SoC-Linux} is only marginally above that of the original \textit{IOb-SoC} and can run in most low-cost FPGAs. The Linux OS takes four minutes and thirty seconds to build. The kernel boots in a \textit{Kintex Ultrascale} device in five seconds and in seven seconds in a \textit{Cyclone V} device. The work developed in this thesis met all the project's goals and went beyond them.
\\
%%
%% Keywords (max 5)
%%
\noindent{{\bf Keywords:}} RISC-V, Linux, Systems on-Chip (SoC), Verilog, IOb-SoC \\

\end{abstract}

